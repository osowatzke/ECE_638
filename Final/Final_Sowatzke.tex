\documentclass[fleqn]{article}
\usepackage[nodisplayskipstretch]{setspace}
\usepackage{amsmath, nccmath, bm}
\usepackage{amssymb}
\usepackage{enumitem}
\usepackage{graphicx}
\usepackage{float}

\newcommand{\zerodisplayskip}{
	\setlength{\abovedisplayskip}{0pt}%
	\setlength{\belowdisplayskip}{0pt}%
	\setlength{\abovedisplayshortskip}{0pt}%
	\setlength{\belowdisplayshortskip}{0pt}%
	\setlength{\mathindent}{0pt}}
	
\newcommand{\norm}[1]{\left \lVert #1 \right \rVert}

\makeatletter
	\newenvironment{equationCenter}{\@fleqnfalse\begin{equation*}}{\end{equation*}}
\makeatother

\title{Final Exam}
\author{Owen Sowatzke}
\date{December 12, 2024}

\begin{document}

	\offinterlineskip
	\setlength{\lineskip}{12pt}
	\setcounter{MaxMatrixCols}{20}
	\zerodisplayskip
	\maketitle
	
	\begin{enumerate}
		\item[2.] An $(n,k)$ linear block code is described by the following parity-check matrix:
		
		\begin{equationCenter}
			\mathbf{H} = \begin{bmatrix}
				0 & 0 & 0 & 0 & 0 & 0 & 0 & 1 & 1 & 1 & 1 & 1 & 1 & 1 & 1 \\
				0 & 0 & 0 & 1 & 1 & 1 & 1 & 0 & 0 & 0 & 0 & 1 & 1 & 1 & 1 \\
				0 & 1 & 1 & 0 & 0 & 1 & 1 & 0 & 0 & 1 & 1 & 0 & 0 & 1 & 1 \\
				1 & 0 & 1 & 0 & 1 & 0 & 1 & 0 & 1 & 0 & 1 & 0 & 1 & 0 & 1
			\end{bmatrix}
		\end{equationCenter}
		
		\begin{enumerate}
			\item Determine the code parameters of this code: codeword length, number of information
bits, code rate, overhead, minimum distance, error correction capability, and error
detection capability.

			\item Represent the $\mathbf{H}$-matrix in systematic form and determine the generator matrix $\mathbf{G}$ of the corresponding systematic code. Provide the corresponding shift register based
encoding circuit of this systematic code.
		\end{enumerate}
	\end{enumerate}
\end{document}