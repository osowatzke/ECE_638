\documentclass[fleqn]{article}
\usepackage[nodisplayskipstretch]{setspace}
\usepackage{amsmath, nccmath, bm}
\usepackage{amssymb}
\usepackage{enumitem}
\usepackage{graphicx}
\usepackage{float}
\usepackage{caption}

\newcommand{\zerodisplayskip}{
	\setlength{\abovedisplayskip}{0pt}%
	\setlength{\belowdisplayskip}{0pt}%
	\setlength{\abovedisplayshortskip}{0pt}%
	\setlength{\belowdisplayshortskip}{0pt}%
	\setlength{\mathindent}{0pt}}
	
\newcommand{\norm}[1]{\left \lVert #1 \right \rVert}

\makeatletter
	\newenvironment{equationCenter}{\@fleqnfalse\begin{equation*}}{\end{equation*}}
\makeatother

\title{Final Exam}
\author{Owen Sowatzke}
\date{December 12, 2024}

\begin{document}

	\offinterlineskip
	\setlength{\lineskip}{12pt}
	\setcounter{MaxMatrixCols}{20}
	\zerodisplayskip
	\maketitle
	
	\begin{enumerate}
		\item In Figure 1 is shown a $2 \times 2$ polarization-time coding MIMO scheme, which
employs dual-polarization transmit and receive antennas. Due to multipath effect, the
initial orthogonality of polarization states is no longer preserved on a receiver side and
we can use the channel coefficients as shown in Fig. 1 to describe this depolarization
effect. Show that $2 \times 2$ scheme can be used to deal with depolarization effect.
Determine the array, diversity and multiplexing gains of this scheme. How would you
determine the channel capacity of this scheme? Consider now a MIMO scheme
employing two dual-polarization Tx antennas and two dual-polarization Rx antennas.
How would approach this system to deal simultaneously with depolarization and
multipath effects? How would you determine the channel capacity of this scheme?

		\begin{figure}[H]
			\centerline{\fbox{\includegraphics[width=0.8\textwidth]{2x2_polarization_time_coding_mimo.png}}}
			\caption{}
			\label{fig::2x2_polarization_time_coding_mimo}
		\end{figure}

		\item[2.] An $(n,k)$ linear block code is described by the following parity-check matrix:
		
		\begin{equationCenter}
			\mathbf{H} = \begin{bmatrix}
				0 & 0 & 0 & 0 & 0 & 0 & 0 & 1 & 1 & 1 & 1 & 1 & 1 & 1 & 1 \\
				0 & 0 & 0 & 1 & 1 & 1 & 1 & 0 & 0 & 0 & 0 & 1 & 1 & 1 & 1 \\
				0 & 1 & 1 & 0 & 0 & 1 & 1 & 0 & 0 & 1 & 1 & 0 & 0 & 1 & 1 \\
				1 & 0 & 1 & 0 & 1 & 0 & 1 & 0 & 1 & 0 & 1 & 0 & 1 & 0 & 1
			\end{bmatrix}
		\end{equationCenter}
		
		\begin{enumerate}
			\item Determine the code parameters of this code: codeword length, number of information
bits, code rate, overhead, minimum distance, error correction capability, and error
detection capability.

			\item Represent the $\mathbf{H}$-matrix in systematic form and determine the generator matrix $\mathbf{G}$ of the corresponding systematic code. Provide the corresponding shift register based
encoding circuit of this systematic code.

			\item The extended code is created by inserting the additional parity-check bits. The most
common way of extending the code is by adding an overall parity-check bit. The extended code is then an $(n+1,k)$ code. Determine the bipartite (Tanner) graph of $\mathbf{H}$-matrix of extended code obtained from the original $\mathbf{H}$-matrix above.

			\item Determine the code parameters of the extended code in (c): codeword length, number of information bits, code rate, overhead, minimum distance, error correction capability, and error detection capability.
		\end{enumerate}
		
		\
	\end{enumerate}
\end{document}