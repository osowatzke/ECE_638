\documentclass[fleqn]{article}
\usepackage[nodisplayskipstretch]{setspace}
\usepackage{amsmath, nccmath, bm}
\usepackage{amssymb}
\usepackage{enumitem}

\newcommand{\zerodisplayskip}{
	\setlength{\abovedisplayskip}{0pt}%
	\setlength{\belowdisplayskip}{0pt}%
	\setlength{\abovedisplayshortskip}{0pt}%
	\setlength{\belowdisplayshortskip}{0pt}%
	\setlength{\mathindent}{0pt}}
	
\makeatletter
	\newenvironment{equationCenter}{\@fleqnfalse\begin{equation*}}{\end{equation*}}
\makeatother

\title{Exam 2}
\author{Owen Sowatzke}
\date{November 12, 2024}

\begin{document}

	\offinterlineskip
	\setlength{\lineskip}{12pt}
	\zerodisplayskip
	\maketitle
	
	\begin{enumerate}
		\item[5.] The space-time code for number of transmit antennas $M_{TX}=4$, number of receive antennas $M_{RX}=1$, and number of channel's uses $N=4$ is described by the following signaling matrix:
		
		\begin{equationCenter}
			\mathbf{X} = \begin{bmatrix}
				x_1 & -x_2^* & -x_3^* &  0 \\ 
				x_2 &  x_1^* &  0     &  x_3^* \\
				x_3 &  0     &  x_1^* & -x_2^* \\
				0   & -x_3^* &  x_2^* &  x_1^*
			\end{bmatrix}
		\end{equationCenter}
		
		\begin{enumerate}
			\item Can this space-time block code be used to design the receiver that will result in spatial interference cancellation? If so describe the corresponding receiver configuration. What is the code rate of this space-time code?
			
			\begin{equation*}
				\mathbf{Y} = \mathbf{H}\mathbf{X} + \mathbf{Z}
			\end{equation*}
			
			\begin{equation*}
				\begin{bmatrix}
					y_1 \\ y_2 \\ y_3 \\ y_4
				\end{bmatrix} = \begin{bmatrix}
					h_1 & h_2 & h_3 & h_4
				\end{bmatrix}\begin{bmatrix}
					x_1 & -x_2^* & -x_3^* &  0 \\ 
					x_2 &  x_1^* &  0     &  x_3^* \\
					x_3 &  0     &  x_1^* & -x_2^* \\
					0   & -x_3^* &  x_2^* &  x_1^*
				\end{bmatrix} + \begin{bmatrix}
					z_1 \\ z_2 \\ z_3 \\ z_4
				\end{bmatrix}
			\end{equation*}
			
			\begin{equation*}
				\begin{bmatrix}
					y_1 \\ y_2 \\ y_3 \\ y_4
				\end{bmatrix} = \begin{bmatrix}
					 h_1x_1   + h_2x_2   + h_3x_3 \\
					-h_1x_2^* + h_2x_1^* - h_4x_3^* \\
					-h_1x_3^* + h_3x_1^* + h_4x_2^* \\
					 h_2x_3^* - h_3x_2^* + h_4x_1^*
				\end{bmatrix} + \begin{bmatrix}
					z_1 \\ z_2 \\ z_3 \\ z_4
				\end{bmatrix}
			\end{equation*}
			
			\begin{equation*}
				\begin{bmatrix}
					y_1 \\ y_2^* \\ y_3^* \\ y_4^*
				\end{bmatrix} = \begin{bmatrix}
					 h_1x_1   + h_2x_2   + h_3x_3 \\
					-h_1^*x_2 + h_2^*x_1 - h_4^*x_3 \\
					-h_1^*x_3 + h_3^*x_1 + h_4^*x_2 \\
					 h_2^*x_3 - h_3^*x_2 + h_4^*x_1
				\end{bmatrix} + \begin{bmatrix}
					z_1 \\ z_2^* \\ z_3^* \\ z_4^*
				\end{bmatrix}
			\end{equation*}
			
			\begin{equation*}
				\begin{bmatrix}
					y_1 \\ y_2^* \\ y_3^* \\ y_4^*
				\end{bmatrix} = \begin{bmatrix}
					h_1   &  h_2   &  h_3 \\
				    h_2^* & -h_1^* & -h_4^* \\
					h_3^* &  h_4^* & -h_1^* \\
					h_4^* & -h_3^* &  h_2^*
				\end{bmatrix} \begin{bmatrix}
					x_1 \\ x_2 \\ x_3
				\end{bmatrix} + \begin{bmatrix}
					z_1 \\ z_2^* \\ z_3^* \\ z_4^*
				\end{bmatrix}
			\end{equation*}
			
			\begin{equation*}
				\mathbf{\tilde{Y}} = \mathbf{\tilde{H}x} + \mathbf{\tilde{z}}
			\end{equation*}
			
			\begin{equation*}
				\mathbf{\tilde{H}^\dagger\tilde{Y}} = \mathbf{\tilde{H}^\dagger\tilde{H}x} + \mathbf{\tilde{H}^\dagger\tilde{z}}
			\end{equation*}
			
			\begin{equation*}
				\mathbf{\tilde{H}^\dagger\tilde{H}} = \begin{bmatrix}
					h_1^* &  h_2 &  h_3 &  h_4 \\
					h_2^* & -h_1 &  h_4 & -h_3 \\
					h_3^* & -h_4 & -h_1 &  h_2
				\end{bmatrix} \begin{bmatrix}
					h_1   &  h_2   &  h_3 \\
				    h_2^* & -h_1^* & -h_4^* \\
					h_3^* &  h_4^* & -h_1^* \\
					h_4^* & -h_3^* &  h_2^*
				\end{bmatrix}
			\end{equation*}
			
			\begin{equation*}
				= \begin{bmatrix}
					|h_1|^2 + \cdots + |h_4|^2 & h_3h_4^* - h_3^*h_4 & h_2^*h_4 - h_2h_4^* \\
					h_3^*h_4 - h_3h_4^* & |h_1|^2 + \cdots + |h_4|^2 & h_1h_4^* - h_1^*h_4 \\
					h_2h_4^* - h_2^*h_4 & h_1^*h_4 - h_1h_4^* & |h_1|^2 + \cdots + |h_4|^2
				\end{bmatrix}
			\end{equation*}
			
			Note that there are multiple terms of the form:
			
			\begin{equation*}
				a - a^* = \text{Re}\{a\} + j\text{Im}\{a\} - (\text{Re}\{a\} - j\text{Im}\{a\}) = j2\text{Im}\{a\}
			\end{equation*}
			
			For the space-time block code to result in spatial interference cancellation, we must be able to write $\mathbf{\tilde{H}^{\dagger}\tilde{H}}$ as $\alpha\mathbf{I}$.
			
			Note that this is not the case in general for the code provided, so it does \textbf{\underline{not}} result in spatial interference cancellation.
			
			However, in a real orthogonal design, the matrix $\mathbf{\tilde{H}}$ is real. This results in $\mathbf{\tilde{H}^\dagger\tilde{H}} = (|h_1|^2 + \cdots + |h_4|^2)\mathbf{I}$. In other words, spatial interference cancellation is realized.
			
			In this case, the coding rate would be $3/4$, because 3 symbols are transmitted in 4 symbol intervals.
		\end{enumerate}
		
	\end{enumerate}
\end{document}